\documentclass[a4paper,oneside,12pt]{extreport}

\include{preamble}
	
\usepackage{amsmath}

\begin{document}

\include{title}

\section{Постановка задачи}
\textbf{Цель работы}. 
Получение навыков разработки алгоритмов решения смешанной краевой
задачи при реализации моделей, построенных на квазилинейном уравнении
параболического типа.

\subsection{Исходные данные}

Задана математическая модель.

Уравнение для функции $T(x,t)$

\begin{equation}
	c(T) \frac{\partial T}{\partial t} = \frac{\partial}{\partial x} (k(T) \frac{\partial T}{\partial x}) - \frac{2}{R}\alpha(x)T + \frac{2T_{0}}{R}\alpha(x)
\end{equation}\\

Краевые условия:

\begin{equation}
	\begin{cases} t = 0, T(x, 0) = T_{0}
	\\ x = 0, - k(T(0)) \frac{\partial T}{\partial x} = F_{0}
	\\ x = l, - k(T(l)) \frac{\partial T}{\partial x} = \alpha_{N} (T(l) - T_{0})
	\end{cases}
\end{equation}\\

В обозначениях уравнения лекции:

\begin{equation}
	p(x) = \frac{2}{R} \alpha(x)
\end{equation}

\begin{equation}
	f(u) = f(x) = \frac{2T_{0}}{R}\alpha(x)
\end{equation}

Разностная схема с разностным краевым условие при $x = 0$:

\begin{equation*}
	(\frac{h}{8} \buildrel\,\,\frown\over{c_{\frac{1}{2}}} + \frac{h}{4} \buildrel\,\,\frown\over{c_{0}} + \buildrel\,\,\frown\over{X_{\frac{1}{2}}} \frac{\tau}{h} + \frac{\tau h}{8} p_{\frac{1}{2}} + \frac{\tau h}{4}p_{0}) \buildrel\,\,\frown\over{y_{0}} + (\frac{h}{8} \buildrel\,\,\frown\over{c_{\frac{1}{2}}} - \buildrel\,\,\frown\over{X_{\frac{1}{2}}} \frac{\tau}{h} + \frac{\tau h}{8} p_\frac{1}{2}) \buildrel\,\,\frown\over{y_{1}} = 
\end{equation*}

\begin{equation}
	= \frac{h}{8} \buildrel\,\,\frown\over{c_{\frac{1}{2}}} (y_{0} + y_{1}) + \frac{h}{4} \buildrel\,\,\frown\over{c_{0}} y_{0} + \buildrel\,\,\frown\over{F}\tau + \frac{\tau h}{4} (\buildrel\,\,\frown\over{f_{\frac{1}{2}}} + \buildrel\,\,\frown\over{c_{0}})
\end{equation}\\

При получении разностного аналога краевого условия при $x = l$ учесть, что поток:

\begin{equation}
	F_{N} = \alpha{N}(y_{N} - T_{0}), F_{N - \frac{1}{2}} = X_{N - \frac{1}{2}} \frac{y_{N - 1} - y_{N}}{h}
\end{equation}

Заданы начальные параметры:\\

\begin{itemize}
	\item $k(T) = a_{1}(b_{1} + c_{1}T^{m_{1}})$, Вт/см K
	\item $c(T) = a_{2} + b_{2} T^{m_{2}} - \frac{c_{2}}{T^2}$, Дж/c$m^3$ K
	\item $a_{1} = 0.0134, b_{1} = 1, c_{1} = 4.35 \cdot 10^{-4}, m_{1} = 1$
	\item $a_{2} = 2.049, b_{2} = 0.563 \cdot 10^{-3}, c_{2} = 0.528 \cdot 10^{5}, m_{2} = 1$
	\item $\alpha(x) = \frac{c}{x - d}, \alpha_{0} = 0.05$ Вт/с$m^2$ K, $\alpha_{N} = 0.01$ Вт/c$m^2$ K
	\item $l = 10$ см
	\item $T_{0} = 300K$
	\item $R$ = 0.5 см
	\item $F(t) = 50$ Вт/с$m^2$
\end{itemize}


\newpage 
\section{Реализация}

\begin{lstlisting}[]
using System;
using System.Collections.Generic;

namespace src
{
	class Conditions
	{
		static public double[] GetLeftConditions(List<double> T)
		{
			double c_plus = Functions.ApproximationPlus(Functions.c, T[0], Constants.t);
			double k_plus = Functions.ApproximationPlus(Functions.k, T[0], Constants.t);

			double K0 = Constants.h / 8 * c_plus + Constants.h / 4 * Functions.c(T[0]) + Constants.t / Constants.h * k_plus + 
					Constants.t * Constants.h / 8 * Functions.p(Constants.h / 2) + Constants.t * Constants.h / 4 * Functions.p(0);

			double M0 = Constants.h / 8 * c_plus - Constants.t / Constants.h * k_plus + Constants.t * Constants.h / 8 * Functions.p(Constants.h / 2);

			double P0 = Constants.h / 8 * c_plus * (T[0] + T[1]) + Constants.h / 4 * Functions.c(T[0]) * T[0] + 
					Constants.F0 * Constants.t + Constants.t * Constants.h / 8 * (3 * Functions.f(0) + Functions.f(Constants.h));

			double[] result = { K0, M0, P0 };
			return result;
		}

		static public double[] GetRightConditions(List<double> T)
		{
			double c_minus = Functions.ApproximationMinus(Functions.c, T[T.Count - 1], Constants.t);
			double k_minus = Functions.ApproximationMinus(Functions.k, T[T.Count - 1], Constants.t);

			double KN = Constants.h / 8 * c_minus + Constants.h / 4 * Functions.c(T[T.Count - 1]) + Constants.t / Constants.h * k_minus + 
					Constants.t * Constants.alphaN + Constants.t * Constants.h / 8 * Functions.p(Constants.l - Constants.h / 2) + 
					Constants.t * Constants.h / 4 * Functions.p(Constants.l);

			double MN = Constants.h / 8 * c_minus - Constants.t / Constants.h * k_minus + 
					Constants.t * Constants.h / 8 * Functions.p(Constants.l - Constants.h / 2);

			double PN = Constants.h / 8 * c_minus * (T[T.Count - 1] + T[T.Count - 2]) + Constants.h / 4 * Functions.c(T[T.Count - 1]) * T[T.Count - 1] + 
					Constants.t * Constants.alphaN * Constants.T0 + Constants.t * Constants.h / 4 * (Functions.f(Constants.l) + Functions.f(Constants.l - Constants.h / 2));

			double[] result = {KN, MN, PN};
			return result;
		}

		static public List<double> GetNewT(List<double> T)
		{
			double[] cond = GetLeftConditions(T);
			double K0 = cond[0], M0 = cond[1], P0 = cond[2];

			cond = GetRightConditions(T);
			double KN = cond[0], MN = cond[1], PN = cond[2];

			List<double> xi = new List<double>();
			xi.Add(0); xi.Add(-M0 / K0);

			List<double> eta = new List<double>();
			eta.Add(0); eta.Add(P0 / K0);

			double x = Constants.h, Tn, denominator, next_xi, next_eta;
			int n = 1;

			while (x + Constants.h < Constants.l)
			{
				Tn = T[n];
				denominator = (Functions.B(x, Tn) - Functions.A(Tn) * xi[n]);

				next_xi = Functions.D(Tn) / denominator;
				next_eta = (Functions.F(x, Tn) + Functions.A(Tn) * eta[n]) / denominator;

				xi.Add(next_xi);
				eta.Add(next_eta);

				n++;
				x += Constants.h;
			} 

			List<double> T_new = new List<double>();
			for (int i = 0; i < n + 1; i++)
				T_new.Add(0);
			
			T_new[n] = (PN - MN * eta[n]) / (KN + MN * xi[n]);
			
			for (int i = n - 1; i > -1; i--)
				T_new[i] = xi[i + 1] * T_new[i + 1] + eta[i + 1];

			return T_new;
		}
	}
}
\end{lstlisting}

\begin{lstlisting}[]
	namespace src
	{
		public static class Constants
		{
			public const double a1 = 0.0134;
			public const double b1 = 1;
			public const double c1 = 4.35e-4;
			public const double m1 = 1;
			public const double a2 = 2.049;
			public const double b2 = 0.563e-3;
			public const double c2 = 0.528e5;
			public const double m2 = 1;
			public const double alpha0 = 0.05;
			public const double alphaN = 0.01;
			public const double l = 10;
			public const double T0 = 300;
			public const double R = 0.5;
			public const double F0 = 50;
			public const double h = 1e-3;
			public const double t = 1;
			public const double eps = 1e-2;
		}
	}
	\end{lstlisting}

\begin{lstlisting}[]
using System;

namespace src
{
	class Functions
	{
		static public double k(double T)
		{
			return Constants.a1 * (Constants.b1 + Constants.c1 * Math.Pow(T, Constants.m1));
		}

		static public double c(double T)
		{
			return Constants.a2 + Constants.b2 * Math.Pow(T, Constants.m2) - (Constants.c2 / Math.Pow(T, 2));
		}
		
		static public double alpha(double x)
		{
			double d = (Constants.alphaN * Constants.l) / (Constants.alphaN - Constants.alpha0);
			double c = -Constants.alpha0 * d;
			return c / (x - d);
		}

		static public double p(double x)
		{
			return alpha(x) * 2 / Constants.R;
		}

		static public double f(double x)
		{
			return alpha(x) * 2 * Constants.T0 / Constants.R;
		}

		static public double ApproximationPlus(Func<double, double> f, double n, double step)
		{
			return (f(n) + f(n + step)) / 2;
		}

		static public double ApproximationMinus(Func<double, double> f, double n, double step)
		{
			return (f(n) + f(n - step)) / 2;
		}

		static public double A(double T)
		{
			return Constants.t / Constants.h * ApproximationMinus(k, T, Constants.t);
		}

		static public double D(double T)
		{
			return Constants.t / Constants.h * ApproximationPlus(k, T, Constants.t);
		}

		static public double B(double x, double T)
		{
			return A(T) + D(T) + Constants.h * c(T) + Constants.h * Constants.t * p(x);
		}

		static public double F(double x, double T)
		{
			return Constants.h * Constants.t * f(x) + T * Constants.h* c(T);
		}


	}
}
\end{lstlisting}

\begin{lstlisting}[]
using System;
using System.Collections.Generic;
using System.Linq;
using System.IO;

namespace src
{
	class Program
	{
		static void SaveData(List<double> res, string fileName)
		{
			FileStream file = new FileStream(fileName, FileMode.Create, FileAccess.Write);
			StreamWriter writer = new StreamWriter(file);

			foreach (var elem in res)
			{
				writer.WriteLine(elem);
			}

			writer.Close();
			file.Close();
		}

		static void Task()
		{
			List<double> T = Enumerable.Range(1, (int)(Constants.l / Constants.h) + 1).Select(x => (double)Constants.T0).ToList();
			List<double> TNew = Enumerable.Range(1, (int)(Constants.l / Constants.h) + 1).Select(x => 0d).ToList();
			List<double> TPrev = new List<double>();

			double currentMax = 1d, ti = 0;

			bool epsilonCondition = true;
			int file_i = 0;
			
			SaveData(T, $"results/data/task_1_{file_i++}.txt");

			while (epsilonCondition)
			{
				TPrev = T;
				currentMax = 1d;

				while (currentMax >= 1)
				{
					TNew = Conditions.GetNewT(T);
					currentMax = Math.Abs((T[0] - TNew[0]) / TNew[0]);

					var arr_d = T.Zip(TNew, (T_i, Tnew_i) => Math.Abs(T_i - Tnew_i) / Tnew_i);
					currentMax = arr_d.Max();

					TPrev = TNew;
				}

				SaveData(TNew, $"results/data/task_1_{file_i++}.txt");

				ti += Constants.t;

				epsilonCondition = false;
				int flag = T.Zip(TNew, (T_i, Tnew_i) => Math.Abs(T_i - Tnew_i) / Tnew_i).Count(elem => elem > Constants.eps);
				if (flag != 0)
					epsilonCondition = true;

				T = TNew;

				// List<double> ys = new List<double>();
				// for (double s = 0; s < Constants.l / 3d> ; s += 0.1)
				// {
				// 	Console.WriteLine(s);
				// 	ys.Add(TNew[(int)(s / Constants.h)]);
				// }
				// SaveData(ys, $"results/data/task_2_{file_i++}.txt");
			}
		}

		static void Main(string[] args)
		{
			// Output with point.
			System.Threading.Thread.CurrentThread.CurrentCulture = new System.Globalization.CultureInfo("en-US");
			Task();
		}
	}
}	
\end{lstlisting}

% \begin{lstlisting}[]
% \end{lstlisting}


\newpage 

\section{Экспериментальная часть}

\section*{Результаты работы программы}

\textbf{1. Представить разностный аналог краевого условия при $x = l$ и его краткий вывод интегро-интерполяционным методом.}\\


Проинтегрируем уравнение на отрезке [$X_{n - \frac{1}{2}}; x_{n}$]. 
Обозначим $F = -k(u) \frac{\partial T}{\partial x}$.

\begin{equation}
    \begin{aligned}
  \int^{x_{N}}_{x_{N-\frac{1}{2}}} dx \int^{t_{m + 1}}_{t_{m}} c(t) \frac{\partial T}{\partial t}dt = - \int^{t_{m + 1}}_{t_{m}}dt \int^{x_{N}}_{x_{N-\frac{1}{2}}} \frac{\partial F}{\partial x} dx - \\
       - \int^{x_{N}}_{x_{N-\frac{1}{2}}} dx \int^{t_{m + 1}}{t_{m}}p(x)Tdt + \int^{x_{N}}_{x_{N-\frac{1}{2}}} dx \int^{t_{m + 1}}{t_{m}} f(x)dt
    \end{aligned}
\end{equation}\\

% \begin{equation}
% 	\int^{x_{N}}_{x_{N-\frac{1}{2}}} \int^{t_{m + 1}}_{t_{m}} c(t) \frac{\partial T}{\partial t}dt = - \int^{t_{m + 1}}_{t_{m}}dt \int^{x_{N}}_{x_{N-\frac{1}{2}}} \frac{\partial F}{\partial x} dx - \int^{x_{N}}_{x_{N-\frac{1}{2}}} dx \int^{t_{m + 1}}{t_{m}}p(x)Tdt + \int^{x_{N}}_{x_{N-\frac{1}{2}}} dx \int^{t_{m + 1}}{t_{m}} f(x)dt
% \end{equation}\\

Интегрируя аналогично разностному аналогу краевого условия при $x = 0$ (из лекции) получим, с учетом \textbf{(6)}:

\begin{equation*}
	\frac{h}{4} (\buildrel\,\,\frown\over{c_{N}}(\buildrel\,\,\frown\over{y_{N}} - y_{N}) - \buildrel\,\,\frown\over{c_{N - \frac{1}{2}}} (\frac{\buildrel\,\,\frown\over{y_{N}} + \buildrel\,\,\frown\over{y_{N - 1}}}{2} - \frac{y_{N} + y_{N + 1}}{2})) =
\end{equation*}

\begin{equation}
	= \tau (\alpha_{N} (\buildrel\,\,\frown\over{y_{N}} - T_{0}) - \buildrel\,\,\frown\over{X_{N}} \frac{\buildrel\,\,\frown\over{y_{N}} + \buildrel\,\,\frown\over{y_{N - 1}}}{h}) - \tau \frac{h}{4} (p_{N}\buildrel\,\,\frown\over{y_{N}} - p_{n - \frac{1}{2} } - \frac{\buildrel\,\,\frown\over{y_{N}} + \buildrel\,\,\frown\over{y_{N - 1}}}{2} + (\buildrel\,\,\frown\over{f_{N}} - \buildrel\,\,\frown\over{f_{N - \frac{1}{2}}}))
\end{equation}\\

Приведем уравнение к виду $K_{N} \buildrel\,\,\frown\over{y_{N}} + M_{N} \buildrel\,\,\frown\over {y_{N - 1}} = P_{N}$:

\begin{equation*}
	(\frac{h}{4} \buildrel\,\,\frown\over{c_{N}} + \frac{h}{8} \buildrel\,\,\frown\over{c_{N - \frac{1}{2}}} + \tau \alpha_{N} + \frac{\tau}{h} \buildrel\,\,\frown\over{X_{N - \frac{1}{2}}} + \frac{h}{4} \tau p_{N} + \frac{h}{8} \tau p_{N - \frac{1}{2}}) \buildrel\,\,\frown\over{y_{n}} + (\frac{h}{8} \buildrel\,\,\frown\over{c_{N - \frac{1}{2}}} - \frac{\tau}{h} \buildrel\,\,\frown\over{X_{N - \frac{1}{2}}} + \frac{h}{8} \tau p_{N - \frac{1}{2}}) \buildrel\,\,\frown\over{y_{N - 1}} =
\end{equation*}

\begin{equation}
	= \alpha_{N} \tau T_{0} + \frac{h}{4} \buildrel\,\,\frown\over{C_{N}}y_{N} + \frac{h}{8} \buildrel\,\,\frown\over{c_{N - \frac{1}{2}}} (y_{N} + y_{N - 1}) + \frac{h}{4} \tau (\buildrel\,\,\frown\over{f_{N}} + \buildrel\,\,\frown\over{f_{N - \frac{1}{2}}})
\end{equation}\\

Будем использовать простую аппроксимацию:

\begin{equation}
	p_{N - \frac{1}{2}} = \frac{p_{N - 1} + p_{N}}{2}
\end{equation}\\

Получим $K_{0}, M_{0}, P_{0}, K_{N}, M_{N}, P_{N}$:

\begin{equation}
	\begin{cases}\buildrel\,\,\frown\over{K_{0}} \buildrel\,\,\frown\over{y_{0}} + \buildrel\,\,\frown\over{M_{0}} \buildrel\,\,\frown\over{y_{1}} = \buildrel\,\,\frown\over{p_{0}}
		\\ \buildrel\,\,\frown\over{A_{n}}\buildrel\,\,\frown\over{y_{n - 1}} - \buildrel\,\,\frown\over{B_{n}} \buildrel\,\,\frown\over{y_{n}} + \buildrel\,\,\frown\over{D_{n}} \buildrel\,\,\frown\over{y_{n + 1}} = - \buildrel\,\,\frown\over{F_{n}}
		\\ \buildrel\,\,\frown\over{K_{n}} \buildrel\,\,\frown\over{y_{N}} + \buildrel\,\,\frown\over{M_{N - 1}} \buildrel\,\,\frown\over{y_{N - 1}} = \buildrel\,\,\frown\over{P_{N}}
	\end{cases}
\end{equation}\\

Систему \textbf{(11)} решим методом итераций ($s$ - номер итерации): 

\begin{equation}
	A^{s - 1}_{n} y^s_{n + 1} - B^{s - 1}_{n} y^{s}_{n} + D^{s - 1}_{n} y^{s}_{n - 1} = -F^{s - 1}_{n}
\end{equation}\\

\textbf{2. График зависимости температуры $T(x, t_{m})$ от координаты $x$ при нескольких фкисированных значених времени $t_{m}$ при заданных выше параметрах. }\\

\begin{figure}[ht!]
	\centering{
		\includegraphics[width=0.7\textwidth]{img/1.png}}
\end{figure}

\textbf{3. График зависимости $T(x_{n}, t)$ при нескольких фиксированных значениях координаты $x_{n}$}\\

На рисунке верхний график соответствует случаю x = 0, нижний случаю x = $l$.

\begin{figure}[ht!]
	\centering{
		\includegraphics[width=0.7\textwidth]{img/2.png}}
\end{figure}

\newpage 

\section{Ответы на вопросы}

\textbf{1. Приведите  результаты тестирования  программы}\\

При отрицательном тепловом потоке слева идет съем тепла.

\begin{figure}[ht!]
	\centering{
		\includegraphics[width=0.7\textwidth]{img/4.png}
		\caption{График зависимости T($x_n$, t) от координаты x при $F_0$ = -10}}
\end{figure}

\begin{figure}[ht!]
	\centering{
		\includegraphics[width=0.7\textwidth]{img/5.png}
		\caption{График зависимости T($x_n$, t) при нескольких фиксированных значениях координаты $x_n$ и $F_0$ = -10}}
\end{figure}

\newpage
При обнулении потока $F_0 (T)$ на выходе получается график температуры, 
установившейся в соответствии с температурой окружающей среды.

\begin{figure}[ht!]
	\centering{
		\includegraphics[width=0.7\textwidth]{img/6.png}}
\end{figure}


\end{document}