\documentclass[a4paper,oneside,12pt]{extreport}

\include{preamble}
	
\usepackage{amsmath}

\begin{document}

\include{title}

\section{Постановка задачи}
\textbf{Цель работы}. 
Получение навыков разработки алгоритмов решения задачи Коши при реализации моделей, построенных на системе ОДУ, 
с использованием метода Рунге-Кутта 4-го порядка точности.

\subsection{Исходные данные}

Задана система электротехнических уравнений, описывающих разрядный контур,
включающий постоянное активное сопротивление $R_k$, нелинейное сопротивление $R_p(I)$,
зависящее от тока $I$, индуктивность $L_k$ и емкость $C_k$

\begin{figure}[ht!]
	\centering{
		\includegraphics[width=0.6\textwidth]{img/1.jpg}
		\caption{Разрядный контур} }
\end{figure}

\begin{equation*}
     \begin{cases}
        \frac{dI}{dT} = \frac{U - (R_k + R_p(I))I}{L_k} \\ 
        \frac{dU}{dt} = -\frac{I}{C_k} \\ 
     \end{cases}
\end{equation*}

Начальные условия:
\begin{equation*}
t=0, I=I_0, U=U_0
\end{equation*}

Здесь $I$, $U$ - ток и напряжение на конденсаторе.

Сопротивление $R_p$ рассчитать по формуле:

\begin{equation*}
R_p = \frac{l_p}{2 \pi R^2 \int_0^1{\sigma(T(z))zdz}}
\end{equation*}


Для функции $T(z)$ применить выражение
$T(z) = T_0 + (T_w - T_0)z^m$

Параметры $T_0$, $m$ находятся интерполяцией из табл.1 
при известном токе $I$.

Коэффициент электропроводности
$\sigma(T)$ зависит от $T$
и рассчитывается интерполяцией из табл.2.
\begin{center}
Таблица 1
\end{center}
\begin{center}
    \begin{tabular}{ |c||c|c| }
        \hline
     $I$, A & $T_0$, К & $m$ \\ 
     \hline
     \hline
     0.5 & 6730 & 0.50 \\  
     \hline
     1 & 6790 & 0.55 \\  
     \hline
     5 & 7150 & 1.7 \\  
     \hline
     10 & 7270 & 3 \\  
     \hline
     50 & 8010 & 11 \\  
     \hline
     200 & 9185 & 32 \\  
     \hline
     400 & 10010 & 40 \\  
     \hline
     800 & 11140 & 41 \\  
     \hline
     1200 & 12010 & 39 \\  
     \hline
     \hline
    \end{tabular}
\end{center}

\begin{center}
    Таблица 2
\end{center}
\begin{center}
    \begin{tabular}{ |c||c| }
        \hline
        $T$, К & $\sigma$, 1/Ом см\\ 
        \hline
        \hline
        4000 & 0.031\\  
        \hline
        5000 & 0.27 \\  
        \hline
        6000 & 2.05 \\  
        \hline
        7000 & 6.06 \\  
        \hline
        8000 & 12.0 \\  
        \hline
        9000 & 19.9 \\  
        \hline
        10000 & 29.6 \\  
        \hline
        11000 & 41.1 \\  
        \hline
        12000 & 54.1 \\  
        \hline
        13000 & 67.7 \\  
        \hline
        14000 & 81.5 \\  
        \hline
        \hline
    \end{tabular}
\end{center}

Параметры разрядного контура:

$R = 0.35$ см

$l_p = 12$ см

$L_k = 187$ $10^{-6}$ Гн

$C_k = 268$ $10^{-6}$ Ф

$R_k = 0.25$ Ом

$U_{co} = 1400$ В

$I_0 = 0...3$ А

$T_w = 2000$ К


\newpage

\section{Теоретическая часть}

\subsection{ОДУ}

Дано ОДУ (Обыкновенное Дифференциальное уравнение) n-ого порядка (\ref{eq:ody}).

\begin{equation}
	F(x, u', u'', ... , u^{(n)} = 0)
	\label{eq:ody}
\end{equation}

ОДУ любого порядка может быть сведено к системе ОДУ 1-ого порядка.

\subsection{Задача Коши}

\textit{Задача Коши} состоит в нахождении решения дифференциального уравнения, удовлетворяющего начальным условиям. 
Это одна из основных задач теории дифференциальных уравнений.

Имеется задача Коши (\ref{eq:ref1}).

\begin{equation}
	{\begin{cases}
			u'(x) = f(x,u) \\
			u(\xi) = \eta
		\end{cases}}
		\label{eq:ref1}
\end{equation}

Методы решения ОДУ в задачи Коши:

\begin{enumerate}
	\item аналитические;
	\item приближенно аналитические;
	\item численные.
\end{enumerate}

\subsection{Метод Рунге-Кутта четвертого порядка точности}

Преимущества схем Р-К.
\begin{enumerate}
    \item Достаточно точные.
    \item Легко изменить шаг.
    \item Методы не требуют перехода к другим методам.
    \item Явные.
\end{enumerate}

Дана система уравнений вида:

\begin{equation}
	{\begin{cases}
			u'(x) = f(x,u,v) \\
			v'(x) = \phi(x,u,v) \\
			u(\xi) = \eta_1 \\
			v(\xi) = \eta_2
		\end{cases}}
		\label{eq:ref1}
\end{equation}

Тогда 

\begin{equation*}
    y_{n+1} = y_n + \frac{k_1+2k_2+2k_3+k_4}{6}
\end{equation*}
    
\begin{equation*}
    z_{n+1} = z_n + \frac{p_1+2p_2+2p_3+p_4}{6}
\end{equation*}

, где

\begin{equation*}
    k_1 = hf(x_n,y_n,z_n)
\end{equation*}

\begin{equation*}
    k_2 = hf(x_n + \frac{h}{2},y_n+\frac{k_1}{2},z_n+\frac{p_1}{2})
\end{equation*}

\begin{equation*}
    k_3 = hf(x_n + \frac{h}{2},y_n+\frac{k_2}{2},z_n+\frac{p_2}{2})
\end{equation*}

\begin{equation*}
    k_4 = hf(x_n + h,y_n+k_3,z_n+p_3)
\end{equation*}


\begin{equation*}
    p_1 = h\phi(x_n,y_n,z_n)
\end{equation*}

\begin{equation*}
    p_2 = h\phi(x_n + \frac{h}{2},y_n+\frac{k_1}{2},z_n+\frac{p_1}{2})
\end{equation*}

\begin{equation*}
    p_3 = h\phi(x_n + \frac{h}{2},y_n+\frac{k_2}{2},z_n+\frac{p_2}{2})
\end{equation*}

\begin{equation*}
    p_4 = h\phi(x_n + h,y_n+k_3,z_n+p_3)
\end{equation*}

\newpage 

\section{Экспериментальная часть}

\subsection{Задание 1}

Графики зависимости от времени импульса t: I(t), U(t), $R_p(t)$, 
произведения I(t) * $R_p(t)$, $T_0(t)$ при заданных выше параметрах.
Шаг сетки: 1e-6

% \begin{figure}[ht!]
% 	\centering{
% 		\includegraphics[width=0.6\textwidth]{img/task1/I.png}
% 		\caption{График зависимости I(t)} }
% \end{figure}
% \begin{figure}[ht!]
% 	\centering{
% 		\includegraphics[width=0.6\textwidth]{img/task1/U.png}
% 		\caption{График зависимости U(t)} }
% \end{figure}
\begin{figure}[ht!]
	\centering{
		\includegraphics[width=0.9\textwidth]{img/task1/sum.png}
		\caption{Графики зависимостей} }
\end{figure}

\begin{figure}[ht!]
	\centering{
		\includegraphics[width=0.4\textwidth]{img/task1/T_0.png}
		\caption{График зависимости $T_0(t)$} }
\end{figure}
% \begin{figure}[ht!]
% 	\centering{
% 		\includegraphics[width=0.6\textwidth]{img/task1/Rp.png}
% 		\caption{График зависимости $R_p(t)$} }
% \end{figure}
% \begin{figure}[ht!]
% 	\centering{
% 		\includegraphics[width=0.6\textwidth]{img/task1/RpI.png}
% 		\caption{График зависимости $R_p*I$} }
% \end{figure}

\newpage
\subsection{Задание 2}

График зависимости I(t) при $R_k + R_p = 0$. 
Колебания тока являются незатухающими.
\begin{figure}[ht!]
	\centering{
		\includegraphics[width=0.6\textwidth]{img/task2/I_2.png}
		\caption{График зависимости I(t) ($R_k + R_p = 0$)} }
\end{figure}

\subsection{Задание 3}

График зависимости I(t) при $R_k + R_p = const = 200$ Ом. 
\begin{figure}[ht!]
	\centering{
		\includegraphics[width=0.6\textwidth]{img/task3/I_3.png}
		\caption{График зависимости I(t) ($R_k + R_p = 200$)} }
\end{figure}

\newpage
\subsection{Задание 4}

Результаты исследования влияния параметров контура
$C_k$, $L_k$, $R_k$ на длительность импульса tимп. апериодической формы.
Длительность импульса определяется по кривой зависимости тока 
от времени на высоте 0.35 *  $I_{max}$  , $I_{max}$ - значение тока в максимуме.

Для исследования влияния параметров контура на длительность импульса tимп
будем использовать приведенный ниже код.

\begin{lstlisting}
    double Imax = arr_I.Max();
    double pulseDuration = arr_I.Count(I => I > Imax * 0.35);
\end{lstlisting}

\subsection{Исследование влияния $C_k$ на длительность импульса.}

Из приведенного исследования видно, что при увеличении $C_k$
длительность импульса увеличивается, а при уменьшении $C_k$
длительность импульса уменьшается.


\begin{figure}[ht!]
	\centering{
		\includegraphics[width=0.6\textwidth]{img/task4/ck_1.png}
		\caption{Исследование влияния $C_k$ на длительность импульса tимп.  } }
\end{figure}

\begin{figure}[ht!]
	\centering{
		\includegraphics[width=0.6\textwidth]{img/task4/ck_2.png}
		\caption{Исследование влияния $C_k$ на длительность импульса tимп.  } }
\end{figure}

\begin{figure}[ht!]
	\centering{
		\includegraphics[width=0.6\textwidth]{img/task4/ck_3.png}
		\caption{Исследование влияния $C_k$ на длительность импульса tимп.  } }
\end{figure}

\begin{figure}[ht!]
	\centering{
		\includegraphics[width=0.6\textwidth]{img/task4/ck_4.png}
		\caption{Исследование влияния $C_k$ на длительность импульса tимп.  } }
\end{figure}

\newpage
\subsection{Исследование влияния $L_k$ на длительность импульса.}

Из приведенного исследования видно, что при увеличении $L_k$
длительность импульса увеличивается, а при уменьшении $L_k$
длительность импульса уменьшается.

\begin{figure}[ht!]
	\centering{
		\includegraphics[width=0.6\textwidth]{img/task4/lk_1.png}
		\caption{Исследование влияния $L_k$ на длительность импульса tимп.  } }
\end{figure}

\begin{figure}[ht!]
	\centering{
		\includegraphics[width=0.6\textwidth]{img/task4/lk_2.png}
		\caption{Исследование влияния $L_k$ на длительность импульса tимп.  } }
\end{figure}

\begin{figure}[ht!]
	\centering{
		\includegraphics[width=0.6\textwidth]{img/task4/lk_3.png}
		\caption{Исследование влияния $L_k$ на длительность импульса tимп.  } }
\end{figure}

\begin{figure}[ht!]
	\centering{
		\includegraphics[width=0.6\textwidth]{img/task4/lk_4.png}
		\caption{Исследование влияния $L_k$ на длительность импульса tимп.  } }
\end{figure}

\newpage
\subsection{Исследование влияния $R_k$ на длительность импульса.}

Из приведенного исследования видно, что при увеличении $R_k$
длительность импульса увеличивается, а при уменьшении $R_k$
длительность импульса уменьшается.

\begin{figure}[ht!]
	\centering{
		\includegraphics[width=0.6\textwidth]{img/task4/rk_1.png}
		\caption{Исследование влияния $R_k$ на длительность импульса tимп.  } }
\end{figure}

\begin{figure}[ht!]
	\centering{
		\includegraphics[width=0.6\textwidth]{img/task4/rk_2.png}
		\caption{Исследование влияния $R_k$ на длительность импульса tимп.  } }
\end{figure}

\begin{figure}[ht!]
	\centering{
		\includegraphics[width=0.6\textwidth]{img/task4/rk_3.png}
		\caption{Исследование влияния $R_k$ на длительность импульса tимп.  } }
\end{figure}

\newpage
\section{Ответы на вопросы}

\textbf{1}. Какие способы тестирования программы, кроме указанного в п.2, можете предложить
ещё?

% Мы моделируем систему систему электротехнических уравнений, описывающих разрядный контур,
Наша программа должна выводить результаты, соответствующие законам физики, 
поэтому можно изучить вид получаемых графиков при стандартных и 
измененных параметрах и сравнить с теоретическим ожидаемым видом.
Также можно протестировать в случае, когда сумма сопротивлений контура равна нулю, 
контур обращается в колебательный, колебания незатухающие.

Также можно тестировать программу при разных значениях шага. 
Малое изменение шага должно приносить малое изменение выходного значения. 

Помимо этого можно сравнивать результаты работы нескольких методов разной точности 
(сравнивать тот метод, который мы использовали с каким-то другим методом).  

\textbf{2}. Получите систему разностных уравнений для решения
сформулированной задачи неявным методом трапеций.
Опишите алгоритм реализации полученных уравнений.

\begin{figure}[ht!]
	\centering{
		\includegraphics[width=0.7\textwidth]{img/trap.jpg}
		\caption{Вопрос 2} }
\end{figure}

После нахождения $I_{n+1}$, используя полученное значение, находится $Uc_{n+1}$. 

\textbf{3}. Из каких соображений проводится выбор численного метода того или иного порядка
точности, учитывая, что чем выше порядок точности метода, тем он более сложен 
и требует, как правило, больших ресурсов вычислительной системы?

На выбор численного метода того или иного порядка влияют требуемая точность результата
и время, которое необходимо для получения результата.
Также на выбор влияет и сама функция.
Если функция сильно изменчива, то стоит выбрать малый шаг вычислений.
Выбор порядка точности метода Рунге-Кутта зависит от порядка производной в правой части - 
стоит использовать метод той же точности.

\textbf{4}. Можно ли метод Рунге-Кутта применить для решения задачи, в которой часть условий
задана на одной границе, а часть на другой?
Например, напряжение по-прежнему задано при t=0, т.е. t=0, U=$U_0$, 
а ток задан в другой момент времени, к примеру, в конце импульса, т.е. при 
t = T, $I=I_T$.  Какой можете предложить алгоритм вычислений?

Можно воспользоваться методом стрельбы и свести краевую задачу к некоторой задаче 
Коши для этой же системы дифференциальных уравнений.
Это позволит применить метод Рунге-Кутта.
В данном случае можно вычислить значение тока в начальный момент
времени или значение напряжения в конечный момент времени.

\end{document}