\documentclass[a4paper,oneside,12pt]{extreport}

\include{preamble}
	
\usepackage{amsmath}

\begin{document}

\include{title}

\section{Постановка задачи}
\textbf{Цель работы}. 
Получение навыков разработки алгоритмов решения краевой задачи при
реализации моделей, построенных на ОДУ второго порядка.

\subsection{Исходные данные}

Задана математическая модель:

\begin{equation*}
	\frac{d}{dx}(\lambda(T)\frac{dT}{dx}) - 4 \cdot k(T) \cdot n_{p}^2 \cdot \sigma \cdot (T^4 - T_{0}^4) = 0
\end{equation*}\\

Краевые условия:

\begin{equation*}
	\begin{cases} x = 0, -\lambda(T(0))\frac{dT}{dx} = F_{0}.
	\\ x = l, -\lambda(T(l))\frac{dT}{dx} = \alpha(T(l) - T_{0})
	\end{cases}
\end{equation*}\\

Функции $\lambda(T)$ и $k(T)$ заданы таблицей.\\

\begin{figure}[ht!]
	\centering{
		\includegraphics[width=0.6\textwidth]{img/table.png}
		\caption{Таблица функций $\lambda(T)$ и $k(T)$ } }
\end{figure}

Заданы начальные параметры:\\
\indent $n_p$ = 1.4 - коэффициент преломления\\
\indent $l$ = 0.2 см - толщина слоя\\
\indent $T_{0}$ = 300К - температура окружающей среды\\
\indent $\sigma$ = 5.668 $\cdot 10^{-12}$ Вт / ($cm^2 \cdot K^4$) - постоянная Стефана - Больцмана\\
\indent $F_{0}$ = 100 Вт / $cm^2$ - поток тепла\\
\indent $\alpha$ = 0.05 Вт / ($cm^2 \cdot K$) - коэффициент теплоотдачи\\

Выход из итераций организовать по температуре и по балансу энергии:

\begin{equation*}
	max|\frac{y^s_n - y^{s-1}_n}{y^s_n}| <= \varepsilon_{1}
\end{equation*} 

\indent для всех $n = 0, 1, ... N.$ и \\

\begin{equation*}
	max|\frac{f^s_1 - f^s_2}{f^s_1}| <= \varepsilon_{1}
\end{equation*}

где \\

\begin{equation*}
	f_{1} = F_0 - \alpha(T(l) - T_{0})
\end{equation*}

\begin{equation*}
	f_{2}  = 4n^2_p \sigma \int^1_0 k(T(x))(T^4(x) - T^4_0) dx
\end{equation*}

\begin{figure}[ht!]
	\centering{
		\includegraphics[width=0.6\textwidth]{img/lab_03.png}
		\caption{Правильное выделение функции p(x) и f(x).} }
\end{figure}

\textbf{Физическое содержание задачи.} 

Сформулированная математическая модель описывает температурное поле T(x)
в плоском слое с внутренними стоками тепловой энергии. Можно представить, что это
стенка из полупрозрачного материала, например, кварца или сапфира, 
нагружаемая тепловым потоком на одной из поверхностей (у нас - слева). Другая поверхность (справа)
охлаждается потоком воздуха, температура которого равна $T_0$. Например, данной схеме
удовлетворяет цилиндрическая оболочка, ограничивающая разряд в газе, т.к. при больших
диаметрах цилиндра стенку можно считать плоской. При высоких температурах раскаленный слой начинает объемно излучать.
Зависимость от температуры излучательной способности материала очень резкая. При низких температурах стенка излучает очень слабо.
Функции lambda(T),k(T ) являются, соответственно, коэффициентами теплопроводности и оптического поглощения материала стенки

\newpage 
\section{Реализация}
\begin{lstlisting}[]
import numpy as np
from scipy import integrate
from scipy.interpolate import InterpolatedUnivariateSpline
import matplotlib.pyplot as plt

class Params:
	F0 = 100
	T0 = 300
	l = 0.2
	h = 1e-3
	np_coef = 1.4
	alpha = 0.05
	sigma = 5.668e-12
	tl_tab = ((300, 500, 800, 1100, 2000, 2400),
		(1.36e-2, 1.63e-2, 1.81e-2, 1.98e-2, 2.50e-2, 2.74e-2))
	tk_tab = ((293, 1278, 1528, 1677, 2000, 2400),
		(2.0e-2, 5.0e-2, 7.8e-2, 1.0e-1, 1.3e-1, 2.0e-1))

params = Params()
get_lambda = InterpolatedUnivariateSpline(params.tl_tab[0], params.tl_tab[1], k=1)
get_k = InterpolatedUnivariateSpline(params.tk_tab[0], params.tk_tab[1], k=1)

def p(T):
	return 0

def f(T):
	return -4 * params.np_coef**2 * params.sigma * get_k(T) * \
					(pow(T, 4)-pow(params.T0, 4))

def LeftConditions(ts):
	h2 = params.h * params.h
	lm = (get_lambda(ts[0]) + get_lambda(ts[1])) / 2
	fm = (f(ts[0]) + f(ts[1])) / 2
	K0 = lm
	M0 = -lm
	P0 = params.h * params.F0 + (h2 / 4) * (f(ts[0]) + fm)
	return K0, M0, P0

def RightConditions(ts):
	KN = get_lambda(ts[-1])
	MN = params.alpha * params.h + get_lambda(ts[-1])
	PN = params.alpha * params.T0 * params.h
	return KN, MN, PN

def RunMethod(A, B, C, D, K0, M0, P0, KN, MN, PN): 
	n = len(A)
	xi = [-M0/K0]
	eta = [P0/K0]
	for i in range(len(A)):
		x = C[i]/(B[i]-A[i] * xi[i])
		e = (D[i]+A[i] * eta[i])/(B[i]-A[i] * xi[i])
		xi.append(x)
		eta.append(e)
	y = [(PN-KN * eta[n])/(MN+KN * xi[n])]
	for i in range(len(A), -1, -1):
		y_i = xi[i] *  y[0]+eta[i]
		y.insert(0, y_i)
	return np.array(y)

def CalcCoefficients(ts):
	K0, M0, P0 = LeftConditions(ts)
	KN, MN, PN = RightConditions(ts)
	N = int(params.l/params.h)
	A, B, C, D = [0]*(N-1), [0]*(N-1), [0]*(N-1), [0]*(N-1)
	for n in range(1, N):
		lam_n_m_1 = get_lambda(ts[n-1])
		lam_n = get_lambda(ts[n])
		lam_n_p_1 = get_lambda(ts[n+1])
		A[n-1] = (lam_n_m_1+lam_n)/2/params.h
		C[n-1] = (lam_n+lam_n_p_1)/2/params.h
		B[n-1] = A[n-1]+C[n-1]+p(ts[n]) * params.h
		D[n-1] = f(ts[n]) * params.h
	return A, B, C, D, K0, M0, P0, KN, MN, PN

def f1(T):
	return params.F0-params.alpha * (T[-1]-params.T0)

def f2 (ts):
	return  4 * params.np_coef**2 * params.sigma * integrate.trapezoid([get_k(ti) * \
				(pow(ti, 4) - pow(params.T0, 4)) for ti in ts], np.arange(0, params.l + params.h, params.h))

def SimpleIterations(alpha, eps1, eps2, maxIter):
	N = int(params.l/params.h)
	cnt = 0 
	T_sp = np.array([params.T0] * (N+1), dtype=float)
	A, B, C, D, K0, M0, P0, KN, MN, PN = CalcCoefficients(T_sp)
	T_s = RunMethod(A, B, C, D, K0, M0, P0, KN, MN, PN)
	while np.max(abs((f1(T_s)-f2(T_s))/f1(T_s))) > eps2 and np.max(abs((T_s-T_sp)/T_s)) > eps1 and cnt < maxIter:
		T_sp = T_s
		A, B, C, D, K0, M0, P0, KN, MN, PN = CalcCoefficients(T_sp)
		new_Ts_norelax = RunMethod(A, B, C, D, K0, M0, P0, KN, MN, PN)
		T_s = T_sp+params.alpha * (new_Ts_norelax-T_sp)
		cnt += 1
	return T_s

def SaveImg(xs, ys, name_x, name_y, file_name):
	fig, ax = plt.subplots()
	ax.plot(xs, ys)
	ax.grid()
	ax.set_xlabel(name_x)
	ax.set_ylabel(name_y)

	# params.alpha *= 3
	# x = np.arange(0, params.l+params.h, params.h)
	# T = SimpleIterations(0.25, 1e-5, 1e-3, 100)
	# ax.plot(x, T)

	plt.savefig(file_name, bbox_inches="tight") # color = 'green'

def main():
	x = np.arange(0, params.l+params.h, params.h)
	T = SimpleIterations(0.25, 1e-5, 1e-3, 100)
	SaveImg(x, T, '1', '2', '1.png')

if __name__ == "__main__":
	main()
\end{lstlisting}

\newpage 

\section{Экспериментальная часть}

\subsection{1. Представить разностный аналог краевого условия при $x = l$ и его краткий вывод интегро-интерполяционным методом.}


\begin{figure}[ht!]
	\centering{
		\includegraphics[width=0.8\textwidth]{img/task_1.jpg}}
\end{figure}


% Проинтегрируем уравнение на отрезке [$X_{n - \frac{1}{2}}; x_{n}$]

% \begin{equation*}
% 	- \int^{x_{n}}_{x_n - \frac{1}{2}} \frac{dF}{dx} dT - \int^{x_n}_{x_n - \frac{1}{2}} P(T) \cdot T^4 dT + \int^{x_{n}}_{x_{n} - \frac{1}{2}} f(t) dT = 0
% \end{equation*}

% Второй и третий интеграл вычислим с помощью метода трапеций:

% \begin{equation*}
% 	F_{n - \frac{1}{2}} - F_{n} - \frac{h}{4} (p_{n} y_{n} + p_{n - \frac{1}{2}}y_{n-\frac{1}{2}}) + \frac{h}{4} (f_{n} + f_{n - \frac{1}{2}}) = 0
% \end{equation*}

% Зная, что: 

% \begin{equation*}
% 	F_{n - \frac{1}{2}} = x_{n - \frac{1}{2}} \frac{y_{n - 1}}{y_n}{h}
% \end{equation*}

% \begin{equation*}
% 	F_{n} = \alpha_{n}(y_{n} - T_{0})
% \end{equation*}

% \begin{equation*}
% 	y_{n - \frac{1}{2}} = \frac{y_{n} + y_{n - 1}}{2h}
% \end{equation*}

% Имеем:

% \begin{equation*}
% 	\frac{x_{n - \frac{1}{2}} y_{n - 1}}{h} - \frac{x_{n - \frac{1}{2}}y_{n}}{h} - \alpha_{n}y_{n} + \alpha_{n} T_{0} - \frac{hp_{n}y_{n}}{48} - \frac{hp_{n - \frac{1}{2}}y_{n}}{8} - \frac{hp_{n - \frac{1}{2}}y_{n - 1}}{8} + \frac{f_{n - \frac{1}{2}} + f_{n}}{4}h = 0
% \end{equation*}

% \begin{equation*}
% 	y_{n}(-\frac{x_{n - \frac{1}{2}}}{h} - \alpha_{n} - \frac{hp_{n}}{4} - \frac{hp_{n} - \frac{1}{2}}{8}) + y_{n - 1}(\frac{x_{n - \frac{1}{2}}}{h} - \frac{hp_{n - \frac{1}{2}}}{8}) = -(\alpha_{n}T_{0} + \frac{f_{n} - \frac{1}{2}}{4}h)
% \end{equation*}

\clearpage

\subsection{2. График зависимости температуры $T(x)$ координаты $x$ при заданных выше параметрах.
Выяснить, как сильно зависят результаты расчета T(x) и необходимое для
этого количество итераций от начального распределения температуры и шага сетки.}

\begin{figure}[ht!]
	\centering{
		\includegraphics[width=0.6\textwidth]{img/1.png}}
\end{figure}

При увеличении $T_0$ общий вид графика не меняется. 
Увеличивается количество итераций.

% Кол-во итераций: 23

% Изменим параметры: $T_0$ = 500. 

% \begin{figure}[ht!]
% 	\centering{
% 		\includegraphics[width=0.6\textwidth]{img/2.png}}
% \end{figure}
% Кол-во итераций: 9

% Начальное распределение температуры не влияет на результат
% При этом количество итераций значительно снижается.

% \newpage

\subsection{3. График зависимости $T(x)$ при $F_{0} = -10 \frac{Bт}{cm^2}$. }


\begin{figure}[ht!]
	\centering{
		\includegraphics[width=0.6\textwidth]{img/3.png}}
\end{figure}

\newpage

\subsection{4. График зависимости $T(x)$ при увеличенных значениях $\alpha$ (например, в 3 раза). Сравнить с п. 2.}

Синяя кривая при увеличенном в 3 раза коэффициенте теплоотдачи.
\begin{figure}[ht!]
	\centering{
		\includegraphics[width=0.6\textwidth]{img/4.png}}
\end{figure}

% Зеленая кривая при увеличенном в 3 раза коэффициенте теплоотдачи.
% \begin{figure}[ht!]
% 	\centering{
% 		\includegraphics[width=0.6\textwidth]{img/5.png}}
% \end{figure}

\newpage

\subsection{5. График зависимости $T(x)$ при $F_{0} = 0$.}

\begin{figure}[ht!]
	\centering{
		\includegraphics[width=0.6\textwidth]{img/6.png} }
\end{figure}

\subsection{6. Для указанного в задании исходного набора параметров привести данные по балансу энергии.}

\begin{itemize}
	\item Точность выхода $\varepsilon_{1} = $ 1e-5 (по температуре)
	\item Точность выхода $\varepsilon_{2} = $ 1e-3 (по балансу)
\end{itemize}


\newpage
\section{Ответы на вопросы}

\textbf{1. Какие способы тестирования программы можно предложить?}\\

В данной программе можно предложить тестирование при помощи изменения значения параметра 
$F_0$. Также при $F_0 > 0$ происходит охлаждение пластины, при $F_0 < 0$ нагревание. При увеличении теплосъема и неизменном потоке
$F_0$ уровень температур T(x) должен снижаться, а градиент увеличиваться.


\textbf{2. Получите  простейший разностный аналог нелинейного краевого условия при $x = l$.}\\

\begin{figure}[ht!]
	\centering{
		\includegraphics[width=0.7\textwidth]{img/v1.jpg}}
\end{figure}


\textbf{3. Опишите алгоритм применения метода прогонки, если при $x = 0$ краевое условие квазилинейное (как в настоящей работе), а при $x = l$, как в п. 2.}\\

\begin{figure}[ht!]
	\centering{
		\includegraphics[width=0.7\textwidth]{img/v2.jpg}}
\end{figure}

\textbf{4. Опишите алгоритм определения единственного значения сеточной функции $y_p$ в одной заданной точке $p$. Использовать встречную прогонку, т.е. комбинацию правой и левой прогонок.}\\

\begin{figure}[ht!]
	\centering{
		\includegraphics[width=0.7\textwidth]{img/v3.jpg}}
\end{figure}

\end{document}